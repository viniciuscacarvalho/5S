\chapter{Implementation}
\section{Hardware and Shell}

\subsection{Project Sections}

\subsection{Task Behavior}

\section{Software}
\subsection{Mobile Communication}
\subsection{AT commands}
\begin{table}[h!]
    \centering
    \begin{tabular}{l|l|l|l}
        \textbf{Field} & \textbf{Value} & \textbf{Meaning} & \textbf{Parameter Description} \\
        \hline
        Run Status & 1 & GNSS engine is running & Indicates whether the GNSS subsystem is powered on (1) or off (0) \\
        Fix Status & (empty) & No valid fix yet & Shows if a location fix is acquired (1 = valid, 0 = invalid) \\
        UTC & (empty) & No time fix yet & Coordinated Universal Time (YYYYMMDDHHMMSS.SSS) \\
        Latitude & 39.500002 & Your current latitude & North/South position in decimal degrees \\
        Longitude & -8.000000 & Your current longitude & East/West position in decimal degrees \\
        Altitude & -54.560 & Meters & Height above mean sea level \\
        Speed & (empty) & Not available yet & Speed over ground in km/h \\
        Course & (empty) & Not available yet & Direction of movement in degrees (0–360°) \\
        Fix Mode & 1 & Autonomous fix (probably 2D/3D) & 1 = Autonomous, 2 = DGPS, 3 = RTK, etc. \\
        HDOP, PDOP, VDOP & 0.1, 0.1, 0.0 & Position accuracy factors & Horizontal, Position, and Vertical dilution of precision \\
    \end{tabular}
    \caption{Detailed Breakdown of +CGNSINF Response Fields}
\end{table}
        
pg 147

separar funções do IMU e GNSS para não atrapalhar um ao outro.
\subsection{DataBase Comunication}
Mongo db \\
JASON