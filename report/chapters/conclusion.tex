\chapter{Conclusion}

The 5S Drifter project almost successfully demonstrated the feasibility of developing a low-power, low-cost, 
and autonomous oceanographic drifter for surface sea stream monitoring. This integrative effort combined
knowledge in embedded systems, wireless communication, sensor integration, and mechanical design to
deliver prototype capable of acquiring GPS position, sea surface temperature and IMU data.

The prototype, works in isolation, but it still needs polishing due to the problems in development.
The physical design of the drifter shell, inspired by successful CMEMS projects such as SONDA and NextSea,
will continue to be improved and tested in sea. 

Overall, the project represents a solid step toward scalable, autonomous marine data collection and 
demonstrates how embedded electronics can be applied effectively in oceanographic research. With future 
iterations and optimizations, the 5S Drifter could become a viable tool for environmental monitoring, 
coastal safety planning, and marine traffic management.


\section{Limitations and Future improvements}
Despite the functionality, the prototype has limitations. Most notably its dependence on battery replacement 
and inability to cover long deployment periods without manual intervention. These issues point to clear 
paths for future improvements, such as integrating solar charging and switching to more energy-efficient 
microcontrollers like the STM32L0 family. A potential full migration to an RTOS-based architecture would 
enhance multitasking and improve energy management.

\section{Final GIT}
\href{https://github.com/viniciuscacarvalho/5S}{Here} is the full project on github.

\section{Special Greetings}
At last, it's important to add the support from the 
CMEMS labs personal as well of the professor Tiago Matos and professor Carlos Faria for the
support with hardware selection and shell mechanical advise respectively.

\begin{thebibliography}{99}

    \bibitem{nazare_canyon}
    Arzola, R. G., Wynn, R. B., Lastras, G., Masson, D. G., \& Weaver, P. P. E. (2008). 
    Different frequencies and triggers of canyon filling and flushing events in Nazare Canyon, offshore Portugal. 
    \textit{Marine Geology}, 250(1-2), 38-63. 
    Available at: \url{https://www.researchgate.net/.../Different_frequencies_and_triggers_of_canyon_filling_and_flushing_events_in_Nazare_Canyon_offshore_Portugal}
    
    \bibitem{secondary_ageostrophic}
    Hernandez-Molina, F. J., Llave, E., \& Somoza, L. (2003). 
    The secondary ageostrophic circulation in the Iberian Poleward Current along the Cantabrian Sea (Bay of Biscay). 
    \textit{Deep-Sea Research Part II}, 50(20-21), 1505-1524.
    
    \bibitem{atmospheric_modes}
    Gonzalez-Pola, C., Lavin, A., \& Vargas-Yanez, M. (2005). 
    Atmospheric modes influence on Iberian Poleward Current variability. 
    \textit{Geophysical Research Letters}, 32(14), L14605.
    
    \bibitem{ocean_waves_energy}
    Green Nation. (n.d.). 
    Ocean Waves - An Energy Source. 
    Retrieved from \url{https://greeneration.org/en/publication/green-info/ocean-waves-energy-source/}
    
    \bibitem{nextsea_lab}
    Matos, T., Rocha, J. L., Martins, M., \& Goncalves, L. M. (2025). 
    Enhancing sea wave monitoring through integrated pressure sensors in smart marine cables. 
    \textit{[Journal/Conference]}, April 2025.
    
    \bibitem{sonda_optical}
    Matos, T., Faria, C. L., Martins, M. S. M., et al. (2022). 
    Development of an automated sensor for in-situ continuous monitoring of sediment deposition and erosion: the SONDA optical instrument. 
    \textit{Science of the Total Environment}, 808, 152164. 
    
    \bibitem{copernicus_svpbrst}
    Poli, P., Lucas, M., O'Carroll, A., Le Menn, M., David, A., Corlett, G. K., Blouch, P., Merchant, C. J., Belbeoch, M., Herklotz, K., et al. (2019). 
    The Copernicus Surface Velocity Platform drifter with Barometer and Reference Sensor for Temperature (SVP-BRST): genesis, design, and initial results. 
    \textit{Ocean Science}, 15, 199-215.
    
    \bibitem{circling_the_seas}
    Subbaraya, S., Breitenmoser, A., Molchanov, A., Muller, J., Oberg, C., Caron, D. A., \& Sukhatme, G. S. (2016). 
    Circling the Seas: Design of Lagrangian Drifters for Ocean Monitoring. 
    \textit{IEEE Robotics \& Automation Magazine}, 23(4), 42-53.
    
    \bibitem{wind_waves_fuel}
    Degraer, S., et al. (2009). 
    Wind and waves affect fuel consumption [Technical report]. Royal Belgian Institute of Natural Sciences.
    
    \bibitem{dala_lora_antenna}
    Dala, A., \& Arslan, T. (2021). 
    Design, Implementation, and Measurement Procedure of Underwater and Water Surface Antenna for LoRa Communication. 
    \textit{Sensors}, 21(4), 1337. doi:10.3390/s21041337
    
    \bibitem{sim_at_commands}
    Wistron NeWeb Corp. (2018, December 4). 
    WNC AT Commands Guide, version 1.2.
    
    \bibitem{wnc_at_commands}
    Wistron NeWeb Corp. (2017, November 17). 
    WNC AT Commands Guide (IMS2 project), rev. 4.1.
    
\end{thebibliography}
    
    