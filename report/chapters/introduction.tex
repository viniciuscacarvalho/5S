\chapter{Project Plan}
    This chapter will briefly talk about the 5S Drifter project motivations as well their function as a product developed 
    by the Minho's University under supervision by the professors Luis Gonçalves and Sérgio Lopes.
\section{Introduction}
Under the course unity of Integrative Project in Industrial Electronics and Computers the students must
apply for professors projects in order to integrate unde their respective laboratories and start to undertand the pace
demanded on the Master's final paper.

This project, given by the professor Luis Gonçalves and Sergio Lopes under the CMEMS laboratory,
has the main porpouse to create a drifter for data aquisition. As a multi-themed project, this report will
explore multiple areas, as the PCB design for hardware and firmware manufacture, software design under the idea to optimize
the execution allowing for better performance. The main goal is to have the final product afloat at the end of the simester.

\begin{figure}[H]
    \centering
    \includegraphics[width=0.7\textwidth]{images/diagrams/shell/unnamed.png}  % Adjust the width as necessary
    \caption{Draft Floater}
    \label{fig:Draft Floater}        
\end{figure}

\subsection{Problem Statement}
The ocean is one of the man greatest mystery even before the written history. Humanity made the world ours over the water, 
from the Portuguese greatests discoveries, braving the raging ocean to the newest oil tanker demanding ever newer technology
in order to tame the sea for safer and smoother sailing.

Nowadays scientists believe only 20\% to 26\% of the ocean is discovered with the actual technology which means that humanity 
know as much about our so grate sky as our own seas. 5S ocean drifter is a equipment made to acquire date from 
superficial sea streams and expand the oceangraphic knowledge about it.

Better knowledge of the ocean lead to further development in diverse areas. Granting safety,
security and efficiency.

5S, an acronym for Sensoring System for Surface Sea Streams is a low-cost, low-power solution to acquire
said data with the focus to last autonomously for the longest time possible. The drifter has to attain its GPS
coordinates in order to track its current and average velocity, alongside with the water temperature and a accelerometer 
information to gather information about the wave intencity. All this data will be stored locally and transmitted by a protocol,
yet to be defined, with a JSON format in order to be recived by a database that already is implemented.  


\subsubsection{Transport}
Sadly, it isn't uncommon to see transport accidents being reported, and even worse, for it to be a gigantic problem.
Some of these accidents are caused by poor mapping of sea conditions, tankers spilling oil, fishing vessels capsizing, leading
to financial problems and even loss of life. Even when there are no accidents, poor knowledge of tides results in higher energy consumption when routes are set against the currents.

A solution would be to create optimized shipping routes, minimizing accidents and improving energy efficiency while 
traversing the waves. Oil tankers could follow currents with lower fuel consumption. Fishing routes could become more
efficient, as their target species may swim with the tides based on temperature and speed. This would ease the workload,
making the activity less reactive and more predictable, aligning expected catch rates with reduced time and energy 
consumption.

A well-known example of a hazardous area is the Nazaré Canyon, where its unique shape creates enormous waves. 
Avoiding these waters is crucial for safer navigation.

\subsubsection{Ecology}


Habitats 

The placement of wave energy converters, a growing field under the energy generation, is one of the main problems the
technology faces. A good positioning improves the efficiency
\begin{figure}
    \centering
    \includegraphics[width=0.7\textwidth]{images/chapter/introduction/renewable_energy.png}  % Adjust the width as necessary
    \caption{The Design of a Wave Energy Converter to Electricity}
    \label{fig:The Design of a Wave Energy Converter to Electricity}        
\end{figure}

Renewable Energy 

\subsubsection{Oceanograpy}
Better undertanding of the Iberian Poleward Current (IPC)

\subsubsection{Geology}

Know where the sedimentation is leading to
\subsubsection{Sports}

\subsection{Problem Statement Analysis}
As a first step into solving this project, an initial construction of the
demends is requested. Here will be presented, following the waterfall aprouch
and UML standarts, the solutions to the individual problems presented by the project.

\subsubsection{Tasks}
The system, in order to acomplish said targets, must set the following topics

\begin{itemize}
    \item Data aquisition
    \begin{itemize}
        \item Power Source Level
        \item Wave intencity
        \item Position
        \item Temperature
    \end{itemize}
    \item Wireless data transferece
    \item Local data storage
    \item Autonomy
    \item Resistant and buotyant shell
\end{itemize}





